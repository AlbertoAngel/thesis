%%%%%%%%%%%%%%%%%%%%%%%%%%%%%%%%%%%%%%%%%%%%%%%%%%%%%%%%%%%%%%%%%%%%%%%%%%%%%%%%%%%%%%%%%%%%%%%%
%%% For print copies
\documentclass[12pt,lot, lof, singlespace]{puthesis}
\newcommand{\printmode}{}

%%% For the electronic copy, use doublespacing, define "\proquestmode" to use outlined links, instead of colored links. 
%\documentclass[12pt,lot, lof]{puthesis}
%\newcommand{\proquestmode}{}
% I prefer proquestmode to be off for electronic copies for normal use, since the colored links are less distracting. However when printed in black and white, the colored links are difficult to read. 

%%% For early drafts without some of the frontmatter
% Also see the "ifodd" command below to disable more frontmatter
%\documentclass[12pt]{puthesis}

%%%%%%%%%%%%%%%%%%%%%%%%%%%%%%%%%%%%%%%%%%%%%%%%%%%%%%%%%%%%%%%%%%%%%%%%%%%%%%%%%%%%%%%%%%%%%%%%
%%%% Author & title page info
\title{Quantum Circuits and their Optical Implementations}
\submitted{May 2012}
\copyrightyear{2012}
\author{Abraham Tibebu Asfaw}
\adviser{Professor Bruce Liby\\Professor Chester Nisteruk}
\departmentprefix{Departments of}
\department{Electrical and Computer Engineering \\and Physics}

%%%%%%%%%%%%%%%%%%%%%%%%%%%%%%%%%%%%%%%%%%%%%%%%%%%%%%%%%%%%%%%%%%%%%%%%%%%%%%%%%%%%%%%%%%%%%%%%
%%%% Tweak float placements

% Alter some LaTeX defaults for better treatment of figures:
    %   General parameters, for ALL pages:
    \renewcommand{\topfraction}{0.85}   % max fraction of floats at top
    \renewcommand{\bottomfraction}{0.6} % max fraction of floats at bottom

    %   Parameters for TEXT pages (not float pages):
    \setcounter{topnumber}{2}
    \setcounter{bottomnumber}{2}
    \setcounter{totalnumber}{4}     % 2 may work better
    \setcounter{dbltopnumber}{2}    % for 2-column pages
    \renewcommand{\dbltopfraction}{0.66}    % fit big float above 2-col. text
    \renewcommand{\textfraction}{0.15}  % allow minimal text w. figs

    %   Parameters for FLOAT pages (not text pages):
    \renewcommand{\floatpagefraction}{0.66} % require fuller float pages
    % N.B.: floatpagefraction MUST be less than topfraction !!
    \renewcommand{\dblfloatpagefraction}{0.66}  % require fuller float pages

%%%%%%%%%%%%%%%%%%%%%%%%%%%%%%%%%%%%%%%%%%%%%%%%%%%%%%%%%%%%%%%%%%%%%%%%%%%%%%%%%%%%%%%%%%%%%%%%
%%%% Use packages

\usepackage{amsmath,amssymb,amsthm,cancel}

%%% For figures
\usepackage{graphicx}
%\usepackage{subfig,rotate}

%%% for comments
\usepackage{verbatim}

%%% For tables
\usepackage{multirow}

% Longtable lets you have tables that span multiple pages.
\usepackage{longtable}

% Booktabs produces far nicer tables than the standard LaTeX tables.
\usepackage{booktabs}

%set parameters for longtable:
% default caption width is 4in for longtable, but wider for normal tables
\setlength{\LTcapwidth}{\textwidth}

%%%%%%%%%%%%%%%%%%%%%%%%%%%%%%%%%%%%%%%%%%%%%%%%%%%%%%%%%%%%%%%%%%%%%%%%%%%%%%%%%%%%%%%%%%%%%%%%
%%% Printed vs. online formatting

\ifdefined\printmode
    % Printed copy
    % url package understands urls (with proper line-breaks) without hyperlinking them
    \usepackage{url}
\else
    \ifdefined\proquestmode
        %ProQuest copy -- http://www.princeton.edu/~mudd/thesis/Submissionguide.pdf
        % ProQuest requires a double spaced version (set previously). They will take an electronic copy, so we want links in the pdf, but also copies may be printed or made into microfilm in black and white, so we want outlined links instead of colored links.
        \usepackage{hyperref}
        \hypersetup{bookmarksnumbered}
        % copy the already-set title and author to use in the pdf properties
        \makeatletter
        \hypersetup{pdftitle=\@title,pdfauthor=\@author}
        \makeatother
    \else
        % Online copy
        % adds internal linked references, pdf bookmarks, etc
        % turn all references and citations into hyperlinks:
        %  -- not for printed copies
        % -- automatically includes url package
        % options:
        %   colorlinks makes links by coloring the text instead of putting a rectangle around the text.
        \usepackage{hyperref}
        \hypersetup{colorlinks,bookmarksnumbered}
        % copy the already-set title and author to use in the pdf properties
        \makeatletter
        \hypersetup{pdftitle=\@title,pdfauthor=\@author}
        \makeatother
        % make the page number rather than the text be the link for ToC entries
        %\hypersetup{linktocpage}
    \fi % proquest or online formatting
\fi % printed or online formatting

%%%%%%%%%%%%%%%%%%%%%%%%%%%%%%%%%%%%%%%%%%%%%%%%%%%%%%%%%%%%%%%%%%%%%%%%%%%%%%%%%%%%%%%%%%%%%%%%
%%%% Define commands

%Highlight notes for future edits to the thesis
%\newcommand{\todo}[1]{\textbf{\emph{TODO:}#1}}

%create an environment that will indent text
%\raggedright makes them left aligned instead of justified

\newenvironment{indenttext}{
    \begin{list}{}{ \itemsep 0in \itemindent 0in
    \labelsep 0in \labelwidth 0in
    \listparindent 0in
    \topsep 0in \partopsep 0in \parskip 0in \parsep 0in
    \leftmargin 1em \rightmargin 0in
    \raggedright
    }
    \item
  }
  {\end{list}}

% another environment that's an indented list, with no spaces between items -- if we want multiple items/lines. Useful in tables. Use \item inside the environment.
%   \raggedright makes them left aligned instead of justified

\newenvironment{indentlist}{
    \begin{list}{}{ \itemsep 0in \itemindent 0in
    \labelsep 0in \labelwidth 0in
    \listparindent 0in
    \topsep 0in \partopsep 0in \parskip 0in \parsep 0in
    \leftmargin 1em \rightmargin 0in
    \raggedright
    }
  }
  {\end{list}}

\newcommand{\ketp}{|\psi\rangle}
\newcommand{\ket}[1]{|#1\rangle}
\newcommand{\tket}[1]{$|#1\rangle$}
\newcommand{\bra}[1]{\langle#1|}
\newcommand{\tbra}[1]{$\langle#1|$}
\newcommand{\braket}[2]{\langle#1|#2\rangle}
\newcommand{\innerp}[2]{\langle#1|#2\rangle}
\newcommand{\tinnerp}[2]{$\langle#1|#2\rangle$}
\newcommand{\outerp}[2]{\q{#1}\qa{#2}}
\newcommand{\touterp}[2]{\tq{#1}\tqa{#2}}
\newcommand{\beq}{\begin{equation}}
\newcommand{\beqnn}{\begin{equation*}}
\newcommand{\eeqnn}{\end{equation*}}
\newcommand{\eeq}{\end{equation}}
\newcommand{\myfig}[4][width=\textwidth]{\begin{figure}\begin{center}\includegraphics[#1]{#2}\caption{#3}\label{#4}\end{center}\end{figure}}
\newcommand{\adj}[1]{#1^\dagger}
\renewcommand{\exp}[1]{\text{exp}\left(#1\right)}
\newcommand{\opH}{\boldsymbol H}
\newcommand{\opN}{\boldsymbol N}
\newcommand{\opp}{\boldsymbol p}
\newcommand{\opx}{\boldsymbol x}
\newcommand{\opP}{\boldsymbol P}
\newcommand{\opX}{\boldsymbol X}
\newcommand{\opa}{\boldsymbol a}
\newcommand{\opaa}{\boldsymbol a^{\dagger}}
\newcommand{\textunder}[2]{\underbrace{#1}_{\hbox{\small #2}}}
\let\oldsqrt\sqrt
\def\sqrt{\mathpalette\DHLhksqrt}
\def\DHLhksqrt#1#2{%
\setbox0=\hbox{$#1\oldsqrt{#2\,}$}\dimen0=\ht0
\advance\dimen0-0.2\ht0
\setbox2=\hbox{\vrule height\ht0 depth -\dimen0}%
{\box0\lower0.4pt\box2}}

%%%%%%%%%%%%%%%%%%%%%%%%%%%%%%%%%%%%%%%%%%%%%%%%%%%%%%%%%%%%%%%%%%%%%%%%%%%%%%%%%%%%%%%%%%%%%%%%
%%%% Front-matter
% For early drafts, you may want to disable some of the frontmatter. Simply change this to "\ifodd 1" to do so.
\ifodd 0
    % front-matter disabled while writing chapters
    \renewcommand{\maketitlepage}{}
    \renewcommand*{\makecopyrightpage}{}
    \renewcommand*{\makeabstract}{}
    % you can just skip the \acknowledgements and \dedication commands to leave out these sections.
\else
    \abstract{
    % Abstract can be any length, but should be max 350 words for a Dissertation for ProQuest's print indicies (150 words for a Master's Thesis) or it will be truncated for those uses.
        Starting with Richard Feynman's suggestion that quantum mechanical computers may be able to efficiently simulate quantum mechanical systems \cite{historyofqc}, there has been a tremendous amount of effort worldwide to create scalable quantum computers. Various experts have analyzed possible implementation schemes for quantum computers. Much like the time when classical computers had several proposed schemes, quantum computers have seen many attractive candidates for their implementation.

Here, we introduce quantum architecture via quantum gates and circuits that result from their combinations. We investigate the mathematical models of these quantum gates and discuss the requirements for scalable quantum circuits. We find universal sets of logic gates that we then implement using optical devices that are commonly available in a laboratory setting. In doing so, we develop the analytical skills necessary for determining the feasibility of quantum architecture schemes. 

Finally, we discuss the drawbacks of the single photon quantum computing scheme and discuss possible areas of research using this scheme.

    }
    \acknowledgements{
        The Manhattan College community has welcomed me with open arms and provided an environment where I was able to discover my academic interests. For this, I am very grateful for this, and for all the support given to me by my friends. 

My attendance at Manhattan College would not have been possible without the generosity of Mr. and Mrs. O'Malley, to whom I am forever indebted. 

As the first engineer in my family, I relied heavily on advice from all the professors in the Department of Electrical and Computer Engineering at Manhattan College. They have gone beyond providing academic mentorship to nurture my varying interests. In particular, Prof. Nisteruk has witnessed my interests as they changed from signals and systems to linear systems, and finally to quantum computing. I am grateful to him for encouraging me to work outside my comfort levels until I had something to be proud of. His frequent lessons of quantum mechanics outside my regular academic coursework have broadened my perspectives beyond the principles of electrical engineering.

Despite my major, I have always received support from the Department of Mathematics and Computer Science. Several professors in the department took the time to give me independent study courses. I am thankful to Prof.\ Bishop, Prof.\ Boothe, Prof.\ Goldstone, Prof.\ Jura and Prof.\ McCabe for their patience and support. In particular, Prof.\ Bishop and Prof.\ Boothe have nurtured my interest in mathematics and computer science through two rewarding research projects that have helped shape my career significantly.

I also want to acknowledge the support provided to me by the Department of Physics, and particularly by Prof. Liby. He took on the role of advising me willingly and filled the gaps in my knowledge of physical optics. His mentorship has prepared me well for my upcoming graduate studies.

I would like to thank my entire family for their continued support. My academic growth has always been supplemented by their wisdom. In particular, I would like to thank my brother Kidus for constantly reviewing anything I sent his way and my mother Hiwot for the emotional support that she has given me throughout my life at and away from home.

I would also like to thank my girlfriend Mahlet for being a constant source of motivation and encouragement.

Last but certainly not least, I would like to thank God for creating me and allowing me to delve into an exploration of the computing capabilities of quantum phenomena afforded by nature.

    }

    \dedication{To R.P. Feynman, whose remarkable imagination brought the world's attention to quantum computers}
\fi  % disable frontmatter

%%%%%%%%%%%%%%%%%%%%%%%%%%%%%%%%%%%%%%%%%%%%%%%%%%%%%%%%%%%%%%%%%%%%%%%%%%%%%%%%%%%%%%%%%%%%%%%%
%%%% Hide some chapters
%%% If you want to produce a pdf that includes only certain chapters, specify them with includeonly, in addition to including all chapters below.
%\includeonly{ch-intro/chapter-intro}
%%% You can also specify multiple chapters.
%\includeonly{ch-intro/chapter-intro,ch-usage/chapter-usage}
%\includeonly{chap1,chap2,chap3}

%%%%%%%%%%%%%%%%%%%%%%%%%%%%%%%%%%%%%%%%%%%%%%%%%%%%%%%%%%%%%%%%%%%%%%%%%%%%%%%%%%%%%%%%%%%%%%%%
%%%% Notes:
% Footnotes should be placed after punctuation.\footnote{place here.}
% Generally, place citations before the period~\cite{anotherauthor}.
% The proper usage for i.e., and e.g., include commas ``(e.g., option A, option B)''

%%%%%%%%%%%%%%%%%%%%%%%%%%%%%%%%%%%%%%%%%%%%%%%%%%%%%%%%%%%%%%%%%%%%%%%%%%%%%%%%%%%%%%%%%%%%%%%%
%%%% Import chapters
\begin{document}

\makefrontmatter

\chapter{Quantum circuits\label{ch:qcirc}}

Classical computers use bits to represent information for computation. A bit can take on only one value at a time from a choice of two permitted values. For example, a bit 1 may be used to encode the information ``turned on'' while the bit 0 may be used to represent ``turned off.'' A string of bits can be used to represent large chunks of information for computation. The study of computer architecture begins with a botton-up approach in which bits are studied first, followed by bit operations leading to more complicated circuitry for manipulating bits. We follow a similar approach here. We will explain the most fundamental components of a quantum computer starting with the most basic element, the qubit.

\section{The qubit}
Qubits are the most fundamental elements of a quantum computer. They correspond to their classical computer counterparts known as bits. The most important difference between qubits and classical bits is the idea of quantum superposition -- while a bit is limited to two values (say 0 and 1), a qubit can be in a state that consists of some component of either 0 and 1. 

The most popular notation for a qubit is through the use of bras and kets following Dirac notation. In this way, the wave function of a qubit is given by

\beq
\label{eq:qubitdef}
\ketp = \alpha\ket{0} + \beta\ket{1}
\eeq
where $\ket{0}$ and $\ket{1}$ are the \textit{computational basis states}. These two states are essentially the quantum analogues of classical bits. In equation \eqref{eq:qubitdef}, $\alpha$ and $\beta$ are complex numbers. 

Classical bits can be measured for their information content without any effect on that information. A classical bit with a value of 1 can be measured to indicate this value. Following the measurement, the bit will remain in the same \textit{state} before measurement. A qubit differs from this significantly. Since the state of a qubit is described by a wave function as shown in \eqref{eq:qubitdef}, its measured value is predicted by quantum mechanics to be either $\ket{0}$ or $\ket{1}$ with probability $|\alpha|^2$ or $|\beta|^2$ respectively. For that reason, the coefficients $\alpha$ and $\beta$ are the \textit{probability amplitudes} of $\ketp$.

Despite this strange behavior of qubits, they exist in nature and can be implemented using several methods. For instance, in chapter \ref{ch:optimp} we realize qubits using photons. Active areas of research in qubit implementation study several schemes including ion traps, electron spins and NMR \cite{nielsen2000}. 

\section{Bloch sphere interpretation of the qubit}
We have discussed the meaning of the probability amplitudes in the qubit wave function defined in \eqref{eq:qubitdef}. Because the probabilities must sum to one, we have the following constraint.
\beq
|\alpha|^2 + |\beta|^2 = 1
\eeq
This results in a geometrical interpretation of the qubit. Because $\alpha$ and $\beta$ are complex numbers, the wave function of a qubit may be rewritten as
\begin{align}
\ketp &= e^{i\gamma}\cos\frac{\theta}{2}\ket{0} + e^{i(\gamma + \varphi)}\sin\frac{\theta}{2}\ket{1}\\
&= e^{i\gamma}\left(\cos\frac{\theta}{2}\ket{0} + e^{i\varphi}\sin\frac{\theta}{2}\ket{1}\right)
\end{align}
for some real numbers $\gamma$, $\theta$ and $\varphi$. Here, note that there is a global phase shift which produces no observable effects. We will ignore it and instead write the qubit as
\beq
\label{eq:gqubitdef}
\ketp = \cos\frac{\theta}{2}\ket{0} + e^{i\varphi}\sin\frac{\theta}{2}\ket{1}
\eeq
\myfig[height=8cm,width=8cm]{qcirc/fig/blochsphere.png}{Bloch sphere interpretation of a qubit}{fig:bsqubit}
This representation of a qubit results in a mapping with the unit sphere. The qubit in \eqref{eq:gqubitdef} is located at the point $(1,\theta,\varphi)$ on the unit sphere in spherical coordinates. This unit sphere is named the \textit{Bloch sphere} after the name of its discoverer. The Bloch sphere is an excellent way of visualizing single qubit states. It is shown in figure \ref{fig:bsqubit}. Sadly, it cannot be used to visualize multiple qubit systems. However, in section \ref{sec:singlequbitops}, we introduce operations on single qubits and visualize the changes that they cause with the help of the Bloch sphere.

\section{Single qubit operations\label{sec:singlequbitops}}
Operations on a qubit must preserve the norm of the qubit, i.e., given an operation $O$ on a single qubit and two qubits $\ket{\psi} = a\ket{0} + b\ket{1}$ and $\ket{\psi'} = O\ket{\psi} = a'\ket{0} + b'\ket{1}$, the normalization conditions
\beq
a^2 + b^2 = a'^2 + b'^2 = 1
\eeq
must hold. For this reason, operators on single qubits are 2x2 unitary matrices. A unitary matrix $U$ has the defining property $\adj{U}U = U\adj{U} = I$ where $\adj{U}$ is the adjoint of $U$. 

The most common single qubit operations are represented by the Pauli matrices.  The Pauli matrices are shown below. 

\beq
X \equiv \left[\begin{array}{cc}0 & 1\\
1 & 0\end{array}\right] \text{ ; } Y \equiv \left[\begin{array}{cc}0 & -i\\
i & 0\end{array}\right] \text{ ; } Z \equiv \left[\begin{array}{cc}1 & 0\\
0 & -1\end{array}\right]
\eeq

Three other matrices that are commonly used in quantum computing are the Hadamard ($H$), $\pi/8$ ($T$) and phase ($S$) gates. These are shown below.
\beq
H \equiv \frac{X+Z}{\sqrt{2}} = \frac{1}{\sqrt{2}}\left[\begin{array}{cc}1 & 1\\
1 & -1\end{array}\right] \text{ ; } T \equiv \left[\begin{array}{cc}1 & 0\\
0 & exp(i\pi/4)\end{array}\right] \text{ ; } S \equiv T^2 = \left[\begin{array}{cc}1 & 0\\
0 & i\end{array}\right]
\eeq
Incidentally, the Hadamard gate is also known as the ``square root of NOT'' gate because it maps $\ket{0}$ to $(\ket{0} + \ket{1})/\sqrt{2}$ and $\ket{0}$ to $(\ket{0} - \ket{1})/\sqrt{2}$, both of which are ``halfway'' between $\ket{0}$ and $\ket{1}$. We can visualize the operation of the Hadamard gate in two steps: 
\begin{enumerate}
\item Rotate the qubit vector in the Bloch sphere about the y axis by 90$^{\circ}$ and
\item Rotate the new qubit vector in the Bloch sphere about the x axis by 180$^{\circ}$
\end{enumerate}
The final vector represents the output of the Hadamard gate.

In Appendix \ref{ch:expmtx}, we have shown how to exponentiate matrices. Using these results, we now introduce three additional unitary matrices known as \textit{rotation matrices} corresponding to the Pauli matrices. These are shown below.

\begin{align}
R_x(\theta) &\equiv \exp{-i\theta X/2} = \cos\left(\frac{\theta}{2}\right)I - i\sin\left(\frac{\theta}{2}\right)X = 
\left[\begin{array}{cc}\cos\frac{\theta}{2} & -i\sin\frac{\theta}{2}\\
-i\sin\frac{\theta}{2} & \cos\frac{\theta}{2}\end{array}\right]\\ 
R_y(\theta) &\equiv \exp{-i\theta Y/2} = \cos\left(\frac{\theta}{2}\right)I - i\sin\left(\frac{\theta}{2}\right)Y = 
\left[\begin{array}{cc}\cos\frac{\theta}{2} & -\sin\frac{\theta}{2}\\
\sin\frac{\theta}{2} & \cos\frac{\theta}{2}\end{array}\right]\\ 
R_z(\theta) &\equiv \exp{-i\theta Z/2} = \cos\left(\frac{\theta}{2}\right)I - i\sin\left(\frac{\theta}{2}\right)Z = 
\left[\begin{array}{cc}\exp{-i\theta/2} & 0\\
0 & \exp{i\theta/2}\end{array}\right]
\end{align}

In general, the rotation by $\theta$ about an axis defined by the real unit vector $\hat{n} = (n_x,n_y,n_z)$ is applied using the following matrix.
\beq
R_{\hat{n}}(\theta) \equiv \exp{-i\theta \frac{n_xX + n_yY + n_zZ}{2}} = \cos\left(\frac{\theta}{2}\right)I - i\sin\left(\frac{\theta}{2}\right)(n_xX + n_yY + n_zZ)
\eeq

The rotation matrices $R_x$, $R_y$ and $R_z$ result in rotations about the $x$, $y$, and $z$ axes respectively on the Bloch sphere. 

We present a very useful way of representing unitary operator matrices below. Any unitary operator $U$ can be represented by a matrix which is a product of rotations in the y and z axes plus a global phase. Appendix \ref{ch:proofunitarytorot} proves this relation.
\begin{align}
\label{eq:unitarytosinglerot}
U &= \exp{i\alpha}R_z(\beta)R_y(\gamma)R_z(\delta)\\ 
&= e^{i\alpha}\left[\begin{array}{cc}e^{-i\beta/2} & 0\\
0 & e^{i\beta/2}\end{array}\right]
\left[\begin{array}{cc}\cos\frac{\gamma}{2} & -\sin\frac{\gamma}{2}\\
\sin\frac{\gamma}{2} & \cos\frac{\gamma}{2}\end{array}\right] 
\left[\begin{array}{cc}e^{-i\delta/2} & 0\\
0 & e^{i\delta/2}\end{array}\right]
\end{align}

Yet another representation of a unitary operator matrix which follows from above is shown below.
\beq
U = \exp{i\alpha}AXBXC
\label{eq:unitarytosingle}
\eeq
The proof follows from \eqref{eq:unitarytosinglerot} by making the following substitutions.
\begin{align}
A &\equiv R_z\left(\beta\right)R_y\left(\frac{\gamma}{2}\right)\\
B &\equiv R_y\left(-\frac{\gamma}{2}\right)R_z\left(-\frac{\delta + \beta}{2}\right)\\
C &\equiv R_z\left(\frac{\delta - \beta}{2}\right)
\end{align}

In the above representation, $A,B$ and $C$ are unitary themselves and $ABC = I$.
These representations will be important as we define controlled operations using multiple qubits in the next section. One additional set of identities that we need to keep in mind is the following. The proofs for these are straightforward substitutions and are not shown here.
\beq
HXH = Z \text{ ; } HYH = -Y \text{ ; } HZH = X
\eeq
Figure \ref{fig:singlequbitgates} summarizes all the single qubit gates and shows their circuit notations.

\myfig[scale=0.50]{qcirc/fig/singlequbitgates.png}{Single-qubit gates and their circuit notations}{fig:singlequbitgates}

\section{Multiple qubits and controlled operations}
So far, we have talked about operations on single qubits. We now discuss a class of families on multiple qubits known as \textit{controlled operations}. In particular, we will discuss two-qubit controlled operations. These discussions can intuitively be generalized for larger numbers of qubits.

\subsection{Multiple qubits}
Before discussing the operations, let us introduce the concept of multiple qubits. We introduced single qubits by discussing their classical counterparts. We saw that given the computational basis composed of two states, the general qubit wavefunction in \eqref{eq:qubitdef} is a superposition of these states with the appropriate probability amplitudes. We will introduce multiple qubits in a similar fashion. Consider two classical qubits. They can be combined to form four classical possibilities: 00, 01, 10 and 11. Measurement will reveal one of these four possible states. Given two qubits, the wave function of their combination is given by
\beq
\ketp = \alpha_{00}\ket{00} + \alpha_{01}\ket{01} + \alpha_{10}\ket{10} + \alpha_{11}\ket{11}
\eeq
such that
\beq
|\alpha_{00}|^2 + |\alpha_{01}|^2 + |\alpha_{10}|^2 + |\alpha_{11}|^2 = 1
\eeq
Following measurement, the wave function will collapse to one of the four states with its respective probability.

Generally, given $n$ qubits, there are $2^n$ states in the computational basis with $2^n$ corresponding probability amplitudes. Therefore, using 500 qubits, we can theoretically encode the state of all atoms in the universe!

\subsection{Controlled operations}
The simplest controlled operation is the controlled version of the classical \textsc{NOT} gate, known as the \textsc{CNOT} gate. It operates on two qubits known as the \textit{control qubit} and the \textit{target qubit}. Its operation is described as follows: if the control qubit is set (to \tket{1}), then the target qubit is inverted (from \tket{0} to \tket{1} and vice versa). The shorthand notation for the \textsc{CNOT} gate is given below.

\beq
\text{\textsc{CNOT}: } \ket{control}\ket{target} \rightarrow \ket{control}\ket{control \oplus target}
\eeq

The matrix representation for the \textsc{CNOT} operator can be determined as follows from the above shorthand notation.
\begin{align}
\text{\textsc{CNOT}: } \ket{0}\ket{0} &\rightarrow \ket{0}\ket{0}\\
\text{\textsc{CNOT}: } \ket{0}\ket{1} &\rightarrow \ket{0}\ket{1}\\
\text{\textsc{CNOT}: } \ket{1}\ket{0} &\rightarrow \ket{1}\ket{1}\\
\text{\textsc{CNOT}: } \ket{1}\ket{1} &\rightarrow \ket{1}\ket{0}
\end{align}

In the computational basis \{\tket{0}, \tket{1}\}, the \textsc{CNOT} gate acts on two qubits and is therefore 4x4. The columns of the \textsc{CNOT} matrix are the outputs of \tket{0}\tket{0}, \tket{0}\tket{1}, \tket{1}\tket{0} and \tket{1}\tket{1} respectively. Hence,

\beq
\textsc{CNOT} \equiv \left[\begin{array}{cccc}1 & 0 & 0 & 0\\
0 & 1 & 0 & 0\\
0 & 0 & 0 & 1\\
0 & 0 & 1 & 0\end{array}\right]
\eeq

\myfig[width=3cm]{qcirc/fig/cnot.png}{Circuit notation for a \textsc{CNOT} gate}{fig:cnot}

The circuit notation for the \textsc{CNOT} operator is shown in figure \ref{fig:cnot}. In general, a controlled-U operator can be represented by a matrix by noting how it affects combinations of input qubits. The shorthand notation is shown below.

\beq
\text{controlled-U: } \ket{control}\ket{target} \rightarrow \ket{control}U^{control}\ket{target}
\eeq

This general notation leads to a general form for the matrix of a controlled-U operator, which is shown below.

\beq
\text{controlled-U} \equiv \left[\begin{array}{cc}I & 0\\
0 & U\end{array}\right]
\eeq

Note that the above matrix is 4x4 because both $I$ and $U$ are 2x2. Also note that the \textsc{CNOT} operator has the same matrix representation as a controlled-X operator. Notationally,

\beq
\textsc{CNOT} \equiv \left[\begin{array}{cccc}1 & 0 & 0 & 0\\
0 & 1 & 0 & 0\\
0 & 0 & 0 & 1\\
0 & 0 & 1 & 0\end{array}\right] = \left[\begin{array}{cc}I & 0\\
0 & X\end{array}\right]
\eeq

The commonly used circuit notation for a controlled-U operation is shown in figure \ref{fig:cugate}.

\myfig{qcirc/fig/controlledu.png}{A controlled-U operator. Note the control and target qubits}{fig:cugate}

Recall from equation \ref{eq:unitarytosingle} that we can represent unitary operators by elementary operators. Therefore, for a controlled-U gate, we have the following: when the control is disabled, the identity matrix is applied to the target qubit. Otherwise, $U = \exp{i\alpha}AXBXC$ is applied to the target qubit. Also recalling the constraints on $A,B$ and $C$ such that $ABC = I$, this implies that the control qubit affects the $X$ operators, turning them into \textsc{CNOT} gates. 

Also note the following.
\begin{align}
\exp{i\alpha}\text{: } \ket{0}\ket{0} &\rightarrow \ket{0}\ket{0}\\
\exp{i\alpha}\text{: } \ket{0}\ket{1} &\rightarrow \ket{0}\ket{1}\\
\exp{i\alpha}\text{: } \ket{1}\ket{0} &\rightarrow \ket{1}\exp{i\alpha}(\ket{0}) = \exp{i\alpha}(\ket{1})\ket{0}\\
\exp{i\alpha}\text{: } \ket{1}\ket{1} &\rightarrow \ket{1}\exp{i\alpha}(\ket{1}) = \exp{i\alpha}(\ket{1})\ket{1}
\end{align}

From the above equations, we observe that the effect of the global phase can be applied to the control or target qubits. Since it only applies when the control qubit is set, we will apply it there, giving the circuit diagram in figure \ref{fig:udecomp}.

\myfig{qcirc/fig/udecomp.png}{A controlled-U gate decomposed into elementary single-qubit gates}{fig:udecomp}

The above discussion easily applies when there are multiple control and target qubits. Given a unitary operator $U$ applied on $n$ control qubits \{$x_1,x_2,\ldots,x_n$\} and $k$ target qubits \{$y_1,y_2,\dots,y_k$\}, we have the following shorthand notation.

\begin{align}
\text{controlled-U: } \ket{x_1}\ket{x_2}\ldots\ket{x_{n-1}}\ket{x_n}\ket{y_1}\ket{y_2}\ldots\ket{y_{k-1}}\ket{y_k} \rightarrow\\ \ket{x_1}\ket{x_2}\ldots\ket{x_{n-1}}\ket{x_n}U^{x_1x_2\ldots x_{n-1}x_n}\ket{y_1}\ket{y_2}\ldots\ket{y_{k-1}}\ket{y_k}
\end{align}

This shorthand notation is shown in circuit notation in figure \ref{fig:cugatem}.

\myfig[height=8cm,width=6cm]{qcirc/fig/controlledumultiple.png}{A multiple-control multiple-target controlled-U operator with 4 control and 3 target qubits}{fig:cugatem}

\section{Summary}

In this chapter, we have introduced the two-state quantum system used in quantum computation known as the qubit. We have indicated its wavefunction and used the Bloch sphere as a means of visualizing qubits. The Bloch sphere is particularly important when we consider single qubit gates. It helps us visualize the effects of the unitary operators that we normally encounter in matrix form.

We have also discussed the fundamental single-qubit gates -- the Pauli gates, their derived rotation operators, the Hadamard, phase and $\pi/8$ gates.

After introducing single-qubit gates, we showed how they can be controlled using additional qubits to form multiple-qubit gates. We showed how to derive the matrices corresponding to these operators and derived general controlled operations in terms of elementary single-qubit gates. The method of control was shown to be applicable even when the number of control and target qubits increases.

For additional reading, including topics such as universal gates and operator approximations, the interested reader is referred to chapter 4 of \cite{nielsen2000}.

\chapter{Quantum mechanical analysis of the harmonic oscillator\label{ch:qmech}}

\section{Motivation}
This chapter will provide the necessary background for understanding optical implementations of quantum gates. We introduce relevant operators such as annihilators and creators based on a quantum mechanical description of the harmonic oscillator. The material in this chapter is based on analysis of the harmonic oscillator provided in \cite{goldstein} and \cite{griffiths}. Additional material concerned with annihilators and creators can be found in \cite{klm}.

\section{Creation and annihilation operators\label{annihilators}}

In linear algebra, the eigenvalues $\Omega$ and eigenvectors $\ket{\Omega}$ of an operator $\boldsymbol O$ are related as follows \cite{linalg}.
\beq
\boldsymbol O\ket{\Omega} = \Omega\ket{\Omega}
\eeq
For real eigenvalues $\Omega$, this essentially means that the effect of the operator $\boldsymbol O$ on its eigenvectors $\ket{\Omega}$ is scaling by a factor of $\Omega$. For complex eigenvalues $\Omega$, the eigenvector $\ket{\Omega}$ is scaled and rotated by the magnitude and phase angle of $\Omega$ respectively. 

In quantum mechanics, physical observables are represented by Hermitian operators that have real eigenvalues \cite{griffiths, linalg}. Recall that a Hermitian operator $\opH$ has the property that $\adj{\opH} = \opH$. We represent operators with boldface symbols throughout the chapter. The Hamiltonian of a one-dimensional harmonic oscillator is given below.
\beq
\opH = \frac{\opp^2}{2m} + \frac{1}{2}K\opx^2
\eeq
In addition, the commutator relationship $[\opx,\opp]$ relates the two physical observables in the above equation. The first term in the Hamiltonian is the kinetic energy, and the second is the potential energy. Thus, the Hamiltonian is the total energy of the system \cite{griffiths}. The normalized position and momentum operators are
\begin{align}
\opX &= \sqrt{\frac{K}{\hbar\omega_0}}\opx\\
\opP &= \frac{\opp}{\sqrt{m\hbar\omega_0}}
\end{align}
where $\omega_0 = \sqrt{\frac{K}{m}}$. The Hamiltonian expressed in terms of these normalized operators becomes
\beq
\label{eq:hamiltonianxp}
\opH = \frac{\hbar\omega_0}{2}\left(\opP^2 + \opX^2\right)
\eeq
with a new commutator relationship $[\opX,\opP] = j$. We introduce the non-Hermitian annihilator $\opa$ and its adjoint $\opaa$ as
\begin{align}
\label{eq:acdef}
\opa &= \frac{1}{\sqrt{2}}\left(\opX + j\opP\right)\\
\opaa &= \frac{1}{\sqrt{2}}\left(\opX - j\opP\right)
\end{align}
with a commutator $[\opa,\opaa]=1$. We are also interested in the anti-commutator $\{\opa,\opaa\}$ of the annihilator and its adjoint. Note that
\begin{align}
\opa\opaa &= \frac{1}{2}\left(\opX^2+\opP^2+1\right)\\
\opaa\opa &= \frac{1}{2}\left(\opX^2+\opP^2-1\right)
\end{align}
Hence, $\{\opa,\opaa\} = \opa\opaa+\opaa\opa = \opP^2 + \opX^2$. The Hamiltonian can now be expressed in terms of this anti-commutator.
\begin{align}
\opH &= \frac{\hbar\omega_0}{2}\left(\opP^2 + \opX^2\right) = \frac{\hbar\omega_0}{2}\left(\opaa\opa + \opa\opaa\right)\\
&= \hbar\omega_0\left(\opaa\opa + \frac{1}{2}\right)
\end{align}
We introduce an additional Hermitian operator known as the number operator $\opN$ such that
\beq
\opN = \opaa\opa
\eeq
that counts the number of energy quanta excited in the harmonic oscillator. The eigenvectors $\ket{n}$ of $\opN$ are known as \textit{Fock} states and the corresponding eigenvalues are denoted as $N_n$ such that
\beq
\opN\ket{n} = N_n\ket{n}
\eeq
Since $\opN$ is Hermitian, it has orthonormal eigenvectors and real eigenvalues. In other words, for two eigenvectors $\ket{n},\ket{m}$ of $\opN$, 
\beq
\braket{m}{n} = \delta_{mn}
\eeq
where $$
\delta_{mn} = \left\{ \begin{array}{rl}
1 &\mbox{ if $m = n$} \\
0 &\mbox{ otherwise}
\end{array} \right.
$$
The annihilation operator and its adjoint have special effects on the Fock states \cite{griffiths}.
\begin{align}
\opa\ket{n} &= \sqrt{n}\ket{n-1}\\
\opaa\ket{n} &= \sqrt{n+1}\ket{n+1}
\end{align}
We define $\opa\ket{0} = 0$ for the ground state of the harmonic oscillator. Note that the application of $\opa$ leads to one less quantum while that of $\opaa$ leads to an additional quantum. For this reason, the two operators are usually called the annihilation and creation operators respectively. Starting from the ground state, we can reach a Fock state $\ket{n}$ by repeated application of the creation operator followed by normalization.
\begin{align}
\ket{n} &= \frac{\opaa\ket{n-1}}{\sqrt{n}} = \frac{\left(\opaa\right)^2\ket{n-2}}{\sqrt{n}\sqrt{n-1}}=\ldots\\
&= \frac{1}{\sqrt{\left(n\right)!}}\left(\opaa\right)^n\ket{0}
\end{align}
Since the ground state has energy $\frac{\hbar\omega_0}{2}$ with each additional quantum contributing the same energy, the eigenvalues of the Hamiltonian of the harmonic oscillator are related to the Hamiltonian as 
\beq
\opH\ket{n} = E_n\ket{n} = \hbar\omega_0\left(n + \frac{1}{2}\right)
\eeq
where $E_n$ is the energy eigenvalue corresponding to the Fock state $\ket{n}$.
\section{Matrix elements of the creation and annihilation operators}
In this section, we will investigate the elements of the matrices that represent the annihilation and creation operators. From \eqref{eq:acdef}, we have
\begin{align}
\label{eq:xaaa}
\opX &= \frac{1}{\sqrt{2}}\left(\opaa + \opa\right)\\
\label{eq:paaa}
\opP &= \frac{j}{\sqrt{2}}\left(\opaa - \opa\right)
\end{align}
For Fock states $\ket{m}$ and $\ket{n}$, we now derive a series of relations that reveal the matrix elements of the annihilation and creation operators. Note that for an operator $\boldsymbol O$, $\bra{m}\boldsymbol O\ket{n}$ is the element $O_{mn}$ of the matrix representation of $\boldsymbol O$.

\begin{align}
\bra{m}\opa\ket{n} &= \sqrt{n}\braket{m}{n-1} = \sqrt{n}\delta_{m,n-1}\\
\bra{m}\opaa\ket{n} &= \sqrt{n+1}\braket{m}{n+1} = \sqrt{n+1}\delta_{m,n+1}\\ 
\bra{m}\opaa\opa\ket{n} &= n\braket{m}{n} = n\delta_{m,n}\\ 
\bra{m}\opa\opaa\ket{n} &= \left(n+1\right)\braket{m}{n} = \left(n+1\right)\delta_{m,n}\\ 
\bra{m}\opX\ket{n} &= \bra{m}\frac{1}{\sqrt{2}}\left(\opaa + \opa\right)\ket{n} = \frac{1}{\sqrt{2}}\left(\sqrt{n+1}\delta_{m,n+1} + \sqrt{n}\delta_{m,n-1}\right)\\ 
\bra{m}\opP\ket{n} &= \bra{m}\frac{j}{\sqrt{2}}\left(\opaa - \opa\right)\ket{n} = \frac{j}{\sqrt{2}}\left(\sqrt{n+1}\delta_{m,n+1} - \sqrt{n}\delta_{m,n-1}\right)\\ 
\bra{m}\opa^2\ket{n} &= \sqrt{n}\bra{m}\opa\ket{n-1} = \sqrt{n(n-1)}\braket{m}{n-2}\notag\\
&= \sqrt{n(n-1)}\delta_{m,n-2}\\ 
\bra{m}\left(\opaa\right)^2\ket{n} &= \sqrt{n+1}\bra{m}\opaa\ket{n+1} = \sqrt{(n+1)(n+2)}\braket{m}{n+2}\notag\\
&= \sqrt{(n+1)(n+2)}\delta_{m,n+2}
\end{align}
Using the above results, the matrices for $\opa$ and $\opaa$ are shown below.
\beq
\opa = \left[\begin{array}{ccccc}0 & \sqrt{1} & 0 & 0 & \ldots\\
0 & 0 & \sqrt{2} & 0 & \ldots\\
0 & 0 & 0 & \sqrt{3} & \ldots\\
\vdots & \vdots & \vdots & \vdots & \ddots\end{array}\right] ~ \opaa = \left[\begin{array}{cccc}0 & 0 & 0 & \ldots\\
\sqrt{1} & 0 & 0 & \ldots\\
0 & \sqrt{2} & 0 & \ldots\\
0 & 0 & \sqrt{3} & \ldots\\
\vdots & \vdots & \vdots & \ddots\end{array}\right]
\eeq
We can now use equations \eqref{eq:hamiltonianxp}, \eqref{eq:xaaa} and \eqref{eq:paaa} to determine the matrix elements of the Hamiltonian operator $H$ as follows.
\begin{align}
\bra{m}\opX^2\ket{n} &= \bra{m}\frac{1}{2}\left(\left(\opaa\right)^2 + \opaa\opa + \opa\opaa + \opa^2\right)\ket{n}\notag\\
&= \frac{1}{2}\left(\sqrt{(n+1)(n+2)}\delta_{m,n+2} + n\delta_{m,n} + (n+1)\delta_{m,n} + \sqrt{n(n-1)}\delta_{m,n-2}\right)\notag\\
&= \frac{1}{2}\left(\sqrt{(n+1)(n+2)}\delta_{m,n+2} + (2n+1)\delta_{m,n} + \sqrt{n(n-1)}\delta_{m,n-2}\right)\\
\bra{m}\opP^2\ket{n} &= \bra{m}-\frac{1}{2}\left(\left(\opaa\right)^2 - \opaa\opa - \opa\opaa + \opa^2\right)\ket{n}\notag\\
&= -\frac{1}{2}\left(\sqrt{(n+1)(n+2)}\delta_{m,n+2} - n\delta_{m,n} - (n+1)\delta_{m,n} + \sqrt{n(n-1)}\delta_{m,n-2}\right)\notag\\
&= -\frac{1}{2}\left(\sqrt{(n+1)(n+2)}\delta_{m,n+2} - (2n+1)\delta_{m,n} + \sqrt{n(n-1)}\delta_{m,n-2}\right)\notag\\
&= \frac{1}{2}\left(-\sqrt{(n+1)(n+2)}\delta_{m,n+2} + (2n+1)\delta_{m,n} - \sqrt{n(n-1)}\delta_{m,n-2}\right)\\
\bra{m}\opH\ket{n} &= \bra{m}\frac{\hbar\omega_0}{2}\left(\opX^2 + \opP^2\right)\ket{n}\notag\\
&= \frac{\hbar\omega_0}{4}\left(\sqrt{(n+1)(n+2)}\delta_{m,n+2} + (2n+1)\delta_{m,n} + \sqrt{n(n-1)}\delta_{m,n-2}\right)\notag\\
&+ \frac{\hbar\omega_0}{4}\left(-\sqrt{(n+1)(n+2)}\delta_{m,n+2} + (2n+1)\delta_{m,n} - \sqrt{n(n-1)}\delta_{m,n-2}\right)
\end{align}
~\hfill$\blacksquare$

\chapter{Optical Implementations\label{ch:optimp}}

\section{What makes a good quantum computer?}

\section{Optical devices for quantum computing}

\section{Quantum gates using optical devices}

\subsection{Single qubit gates}

\subsection{Multiple qubit gates}

\section{Drawbacks of the optical scheme}


\chapter{Conclusion\label{ch:conc}}

In this thesis, we have achieved three objectives. First, we introduced the fundamental elements of quantum computation known as qubits. Then, we defined operations on individual qubits and discussed a universal set of operations that can be combined to form all others. We then showed operations that act on two qubits. The objective of this was to eventually find a universal set of operations consisting of single and multiple qubit gates that can be combined to form larger quantum circuits.

Next, we showed a quantum mechanical analysis of the harmonic oscillator as a preparation for following chapters. We discussed annihilator and creator operators in detail and determined the elements of their matrix representations.

Our next task was to realize these quantum gate operations. We discussed what makes a good quantum computer and showed some quantum computation schemes that have gained popularity through many years of research. We focused on the optical scheme in particular, and showed how to implement single qubit gates using retarders, beamsplitters and phase shifters. In order to realize multiple qubit gates, we introduced non-linearity through the use of Kerr media whose indices of refraction are related to the intensities of their inputs. We realized the controlled NOT gate using Kerr media and concluded by noting that single qubit gates and the controlled NOT gate form a universal set.

Finally, we dicussed the drawbacks of the optical scheme of quantum computation in preparation for further research on other schemes. We highlighted the weakness in our scheme resulting from the poor cross phase modulation of Kerr media and introduced cavity quantum electrodynamics as a scheme that attempts to solve this problem.

Our mathematical toolkit was strengthened with several figures and appendices that proved formulae used throughout the discussion.

Additional material on quantum computation can be found in \cite{nielsen2000}. The necessary background of quantum mechanics can be found in various sections throughout \cite{griffiths}. Optical implementations beyond the ones discussed here are given in a groundbreaking paper on linear optics quantum computing (LOQC) \cite{klm} and summarized in \cite{myers-2005,knill-2000,knill-2002,knill-2002-postsel}. General rules for physical implementations of quantum computation are given in \cite{divincenzo}. The fundamentals of optics are discussed in \cite{hechtoptics} and expanded to quantum optics in \cite{foxqoptics}. The necessary background in linear algebra may be obtained from \cite{linalg}. For the mathematically inclined, \cite{liealgs,liealgs2,liealgs3} relate the mathematical ideas of quantum computing to Lie algebras. 


\appendix % all chapters following will be labeled as appendices
\chapter{Exponentiating Matrices\label{ch:expmtx}}

In this appendix, we prove the following.
\beq
\exp{iAx} = \cos(x)I + i\sin(x)A
\eeq
for a real number $x$ and matrix $A$ such that $A^2 = I$ and $A^0 = I$. Recall the power series expansion for the exponential $\exp{x}$ for all $x$.

\beq
\exp{x} = \sum\limits_{n=0}^{\infty} \frac{x^n}{n!}
\eeq
We now rewrite this power series expansion for $\exp{x}$ after replacing $x$ by $iAx$.

\begin{align}
\exp{iAx} &= \sum\limits_{n=0}^{\infty}\frac{\left(iAx\right)^n}{n!}\\
&= \frac{I}{0!} + \frac{iAx}{1!} + \frac{\left(iAx\right)^2}{2!} + \frac{\left(iAx\right)^3}{3!} + \ldots\\
&= \frac{A^2}{0!} + \frac{iAx}{1!} + \frac{\left(iAx\right)^2}{2!} + \frac{\left(iAx\right)^3}{3!} + \ldots
\end{align}
Noting that even powers of $A$ reduce to identity and $i^2 = -1$, we now rearrange the terms in the above equation as follows.

\begin{align}
\exp{iAx} &= \frac{A^2}{0!} + \frac{iAx}{1!} + \frac{\left(iAx\right)^2}{2!} + \frac{\left(iAx\right)^3}{3!} + \ldots\\
&= \left(\frac{x^0}{0!} - \frac{x^2}{2!} + \frac{x^4}{4!} - \ldots \right)I + i\left(\frac{x^1}{1!} - \frac{x^3}{3!} + \frac{x^5}{5!} - \ldots \right)A\\
\end{align}
The power series expansions for the sine and cosine functions appear in the above equation. We will state the power series expansions for these two functions below.

\begin{align}
\sin x &= \sum\limits_{n=0}^{\infty} \left(-1\right)^n\frac{x^{2n+1}}{\left(2n+1\right)!} = \left(\frac{x^1}{1!} - \frac{x^3}{3!} + \frac{x^5}{5!} - \ldots \right)\\
\cos x &= \sum\limits_{n=0}^{\infty} \left(-1\right)^n\frac{x^{2n}}{\left(2n\right)!} = \left(\frac{x^0}{0!} - \frac{x^2}{2!} + \frac{x^4}{4!} - \ldots \right)
\end{align}
Therefore,

\begin{align}
\exp{iAx} &= \left(\frac{x^0}{0!} - \frac{x^2}{2!} + \frac{x^4}{4!} - \ldots \right)I + i\left(\frac{x^1}{1!} - \frac{x^3}{3!} + \frac{x^5}{5!} - \ldots \right)A\\
&= \cos(x)I + i\sin(x)A\qed
\end{align}

\chapter{Decomposing Unitary Matrices into Rotations\label{ch:proofunitarytorot}}

In this appendix, we prove that any unitary 2x2 matrix $U$ can be decomposed into rotations as defined in \eqref{eq:unitarytosinglerot}, which we repeat here for convenience.

\begin{align}
\label{eq:toprove}
U &= \exp{i\alpha}R_z(\beta)R_y(\gamma)R_z(\delta)\\ 
&= e^{i\alpha}\left[\begin{array}{cc}e^{-i\beta/2} & 0\\
0 & e^{i\beta/2}\end{array}\right]
\left[\begin{array}{cc}\cos\frac{\gamma}{2} & -\sin\frac{\gamma}{2}\\
\sin\frac{\gamma}{2} & \cos\frac{\gamma}{2}\end{array}\right] 
\left[\begin{array}{cc}e^{-i\delta/2} & 0\\
0 & e^{i\delta/2}\end{array}\right]
\end{align}

Consider the unitary matrix below. It is formed by carrying out the product above.
\beq
U = \left[\begin{array}{cc}e^{i\left(\alpha-\frac{\beta}{2}-\frac{\delta}{2}\right)}\cos\frac{\gamma}{2} & -e^{i\left(\alpha-\frac{\beta}{2}+\frac{\delta}{2}\right)}\sin\frac{\gamma}{2}\\
e^{i\left(\alpha+\frac{\beta}{2}-\frac{\delta}{2}\right)}\sin\frac{\gamma}{2} & e^{i\left(\alpha+\frac{\beta}{2}+\frac{\delta}{2}\right)}\cos\frac{\gamma}{2}\end{array}\right]
\eeq
We claim that this is a unitary matrix for real numbers $\alpha$, $\beta$, $\gamma$ and $\delta$. In order to prove this, we first find the adjoint of $U$, which is simply the conjugate transpose.
\beq
\adj{U} = \left[\begin{array}{cc}e^{-i\left(\alpha-\frac{\beta}{2}-\frac{\delta}{2}\right)}\cos\frac{\gamma}{2} & e^{-i\left(\alpha+\frac{\beta}{2}-\frac{\delta}{2}\right)}\sin\frac{\gamma}{2}\\
-e^{-i\left(\alpha-\frac{\beta}{2}+\frac{\delta}{2}\right)}\sin\frac{\gamma}{2} & e^{-i\left(\alpha+\frac{\beta}{2}+\frac{\delta}{2}\right)}\cos\frac{\gamma}{2}\end{array}\right]
\eeq
Now, we compute $U\adj{U}$.
\begin{align}
U\adj{U} &= \left[\begin{array}{cc}e^{i\left(\alpha-\frac{\beta}{2}-\frac{\delta}{2}\right)}\cos\frac{\gamma}{2} & -e^{i\left(\alpha-\frac{\beta}{2}+\frac{\delta}{2}\right)}\sin\frac{\gamma}{2}\\
e^{i\left(\alpha+\frac{\beta}{2}-\frac{\delta}{2}\right)}\sin\frac{\gamma}{2} & e^{i\left(\alpha+\frac{\beta}{2}+\frac{\delta}{2}\right)}\cos\frac{\gamma}{2}\end{array}\right]\left[\begin{array}{cc}e^{-i\left(\alpha-\frac{\beta}{2}-\frac{\delta}{2}\right)}\cos\frac{\gamma}{2} & e^{-i\left(\alpha+\frac{\beta}{2}-\frac{\delta}{2}\right)}\sin\frac{\gamma}{2}\\
-e^{-i\left(\alpha-\frac{\beta}{2}+\frac{\delta}{2}\right)}\sin\frac{\gamma}{2} & e^{-i\left(\alpha+\frac{\beta}{2}+\frac{\delta}{2}\right)}\cos\frac{\gamma}{2}\end{array}\right]\\
&= \left[\begin{array}{cc}\cos^2\frac{\gamma}{2}+\sin^2\frac{\gamma}{2} & e^{-i\beta}\cos\frac{\gamma}{2}\sin\frac{\gamma}{2}-e^{-i\beta}\sin\frac{\gamma}{2}\cos\frac{\gamma}{2}\\
e^{i\beta}\sin\frac{\gamma}{2}\cos\frac{\gamma}{2}-e^{i\beta}\cos\frac{\gamma}{2}\sin\frac{\gamma}{2} & \sin^2\frac{\gamma}{2}+\cos^2\frac{\gamma}{2}\end{array}\right]\\
&= \left[\begin{array}{cc}1 & 0\\
0 & 1\end{array}\right]\\
&= I
\end{align}
Hence, $U\adj{U} = I$. In other words, $\adj{U} = U^{-1}$. This is exactly the definition of a unitary matrix. Recalling that $U$ was made up of the three smaller matrices with a global phase in \eqref{eq:toprove} we have shown that any 2x2 unitary matrix can be written as a product of rotations in the z and y axes plus a global phase shift.

More generally, for two non-parallel real unit vectors $\hat{m}$ and $\hat{n}$, one may write
\beq
U = e^{i\alpha}R_{\hat{n}}(\beta)R_{\hat{m}}(\gamma)R_{\hat{n}}(\delta)
\eeq
for some $\alpha$, $\beta$, $\gamma$ and $\delta$.

\chapter{The Baker-Campbell-Hausdorf Formula\label{ch:bchf}}
The Baker-Campbell-Hausdorf formula for a complex number $\lambda$ and operators $A,G$ and $C_n$ is given by
\beq
\label{eq:bchf}
e^{\lambda G}Ae^{-\lambda G} = \sum\limits_{n = 0}^\infty \frac{\lambda^n}{n!} C_n
\eeq
where $C_n$ is defined recursively such that $C_0 = A$ and $C_n = [G,C_{n-1}]$. Several references on Lie Algebras such as \cite{liealgs}, \cite{liealgs2} and \cite{liealgs3} provide interesting proofs of the formula.

In this paper, we are interested in applying the Baker-Campbell-Hausdorf formula to the unitary matrix which represents the operation of an optical beamsplitter. In particular, for a beamsplitter $B$ of angle $\theta$, the unitary matrix representation is
\beq
\label{eq:bseq}
B = \exp{\theta(\adj{a} b - a\adj{b})}
\eeq
For such a beamsplitter, we will compute $Ba\adj{B}$ and $Bb\adj{B}$ for annihilation operators $a$ and $b$ and creation operators $\adj{a}$ and $\adj{b}$. Recall from \ref{annihilators} that for annihilation operators $\alpha$ and $\beta$, $[\alpha,\adj{\alpha}] = [\beta,\adj{\beta}] = 1$ and $[\alpha,\beta] = [\adj{\alpha},\beta] = [\alpha,\adj{\beta}] = [\adj{\alpha},\adj{\beta}] = 0$. 
Setting $G$ in \eqref{eq:bchf} to $\adj{a} b - a\adj{b}$, we have
\begin{align*}
[G,a] &= (\adj{a} b - a\adj{b})a - a(\adj{a} b - a\adj{b}) \\
&= \adj{a}ba - a\adj{b}a - a\adj{a}b + aa\adj{b}\\
&= \adj{a}ba - a\adj{a}b + aa\adj{b} - a\adj{b}a\\
&= [\adj{a}b,a] + a\cancelto{0}{[a,\adj{b}]}\\
&= [\adj{a}b,a]\\
&= \adj{a}ba - \adj{a}ab + \adj{a}ab - a\adj{a}b\\
&= \adj{a}\cancelto{0}{[b,a]} - [a,\adj{a}]b\\
&= -b
\end{align*}
Hence, 
\begin{align}
\label{eq:Gcommutes}
[G,a] &= -b\\
[G,b] &= a
\end{align}
We can now generalize the terms of $C_n$. Setting $C_0 = a$, $C_1 = [G,a] = -b$, $C_2 = [G,C_1] = -a$, $C_3 = [G,C_2] = b,\ldots$ and generally,
\begin{align*}
C_n &= i^na \text{ for even n}\\
C_n &= i^{n+1}b \text{ for odd n}
\end{align*}
Hence,
\begin{align*}
Ba\adj{B} &= e^{\theta G}ae^{-\theta G}\\
&= \sum\limits_{n=0}^\infty \frac{\theta^n}{n!}C_n\\
&= \sum\limits_{n = even} \frac{(i\theta)^n}{n!}a + i\sum\limits_{n = odd}^\infty \frac{(i\theta)^n}{n!}b\\
&= a\cos\theta + b\sin\theta
\end{align*}
Similarly, $Bb\adj{B} = -a\sin\theta + b\cos\theta$.



%\include{appendicies/}

\singlespacing
\bibliographystyle{plain}
\cleardoublepage
\ifdefined\phantomsection
    \phantomsection
\else
\fi
\addcontentsline{toc}{chapter}{Bibliography}
\bibliography{thesis}
\end{document}
