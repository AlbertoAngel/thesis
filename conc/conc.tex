\chapter{Conclusion\label{ch:conc}}

In this thesis, we have achieved three objectives. First, we introduced the fundamental elements of quantum computation known as qubits. Then, we defined operations on individual qubits and discussed a universal set of operations that can be combined to form all others. We then showed operations that act on two qubits. The objective of this was to eventually find a universal set of operations consisting of single and multiple qubit gates that can be combined to form larger quantum circuits.

Next, we showed a quantum mechanical analysis of the harmonic oscillator as a preparation for following chapters. We discussed annihilator and creator operators in detail and determined the elements of their matrix representations.

Our next task was to realize these quantum gate operations. We discussed what makes a good quantum computer and showed some quantum computation schemes that have gained popularity through many years of research. We focused on the optical scheme in particular, and showed how to implement single qubit gates using retarders, beamsplitters and phase shifters. In order to realize multiple qubit gates, we introduced non-linearity through the use of Kerr media whose indices of refraction are related to the intensities of their inputs. We realized the controlled NOT gate using Kerr media and concluded by noting that single qubit gates and the controlled NOT gate form a universal set.

Finally, we dicussed the drawbacks of the optical scheme of quantum computation in preparation for further research on other schemes. We highlighted the weakness in our scheme resulting from the poor cross phase modulation of Kerr media and introduced cavity quantum electrodynamics as a scheme that attempts to solve this problem.

Our mathematical toolkit was strengthened with several figures and appendices that proved formulae used throughout the discussion.

Additional material on quantum computation can be found in \cite{nielsen2000}. The necessary background of quantum mechanics can be found in various sections throughout \cite{griffiths}. Optical implementations beyond the ones discussed here are given in a groundbreaking paper on linear optics quantum computing (LOQC) \cite{klm} and summarized in \cite{myers-2005,knill-2000,knill-2002,knill-2002-postsel}. General rules for physical implementations of quantum computation are given in \cite{divincenzo}. The fundamentals of optics are discussed in \cite{hechtoptics} and expanded to quantum optics in \cite{foxqoptics}. The necessary background in linear algebra may be obtained from \cite{linalg}. For the mathematically inclined, \cite{liealgs,liealgs2,liealgs3} relate the mathematical ideas of quantum computing to Lie algebras. 
