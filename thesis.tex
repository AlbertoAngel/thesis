%%%%%%%%%%%%%%%%%%%%%%%%%%%%%%%%%%%%%%%%%%%%%%%%%%%%%%%%%%%%%%%%%%%%%%%%%%%%%%%%%%%%%%%%%%%%%%%%
%%% For print copies
%\documentclass[12pt,lot, lof, singlespace]{puthesis}
%\newcommand{\printmode}{}

%%% For the electronic copy, use doublespacing, define "\proquestmode" to use outlined links, instead of colored links. 
\documentclass[12pt,lot, lof]{puthesis}
\newcommand{\proquestmode}{}
% I prefer proquestmode to be off for electronic copies for normal use, since the colored links are less distracting. However when printed in black and white, the colored links are difficult to read. 

%%% For early drafts without some of the frontmatter
% Also see the "ifodd" command below to disable more frontmatter
%\documentclass[12pt]{puthesis}

%%%%%%%%%%%%%%%%%%%%%%%%%%%%%%%%%%%%%%%%%%%%%%%%%%%%%%%%%%%%%%%%%%%%%%%%%%%%%%%%%%%%%%%%%%%%%%%%
%%%% Author & title page info
\title{Quantum Circuits and their Optical Implementations}
\submitted{May 2012}
\copyrightyear{2012}
\author{Abraham Tibebu Asfaw}
\adviser{Professor Bruce Liby\\Professor Chester Nisteruk}
\departmentprefix{Department of}
\department{Electrical and Computer Engineering}

%%%%%%%%%%%%%%%%%%%%%%%%%%%%%%%%%%%%%%%%%%%%%%%%%%%%%%%%%%%%%%%%%%%%%%%%%%%%%%%%%%%%%%%%%%%%%%%%
%%%% Tweak float placements

% Alter some LaTeX defaults for better treatment of figures:
    %   General parameters, for ALL pages:
    \renewcommand{\topfraction}{0.85}   % max fraction of floats at top
    \renewcommand{\bottomfraction}{0.6} % max fraction of floats at bottom

    %   Parameters for TEXT pages (not float pages):
    \setcounter{topnumber}{2}
    \setcounter{bottomnumber}{2}
    \setcounter{totalnumber}{4}     % 2 may work better
    \setcounter{dbltopnumber}{2}    % for 2-column pages
    \renewcommand{\dbltopfraction}{0.66}    % fit big float above 2-col. text
    \renewcommand{\textfraction}{0.15}  % allow minimal text w. figs

    %   Parameters for FLOAT pages (not text pages):
    \renewcommand{\floatpagefraction}{0.66} % require fuller float pages
    % N.B.: floatpagefraction MUST be less than topfraction !!
    \renewcommand{\dblfloatpagefraction}{0.66}  % require fuller float pages

%%%%%%%%%%%%%%%%%%%%%%%%%%%%%%%%%%%%%%%%%%%%%%%%%%%%%%%%%%%%%%%%%%%%%%%%%%%%%%%%%%%%%%%%%%%%%%%%
%%%% Use packages

\usepackage{amsmath,amssymb,amsthm}

%%% For figures
\usepackage{graphicx}
%\usepackage{subfig,rotate}

%%% for comments
\usepackage{verbatim}

%%% For tables
\usepackage{multirow}

% Longtable lets you have tables that span multiple pages.
\usepackage{longtable}

% Booktabs produces far nicer tables than the standard LaTeX tables.
\usepackage{booktabs}

%set parameters for longtable:
% default caption width is 4in for longtable, but wider for normal tables
\setlength{\LTcapwidth}{\textwidth}

%%%%%%%%%%%%%%%%%%%%%%%%%%%%%%%%%%%%%%%%%%%%%%%%%%%%%%%%%%%%%%%%%%%%%%%%%%%%%%%%%%%%%%%%%%%%%%%%
%%% Printed vs. online formatting

\ifdefined\printmode
    % Printed copy
    % url package understands urls (with proper line-breaks) without hyperlinking them
    \usepackage{url}
\else
    \ifdefined\proquestmode
        %ProQuest copy -- http://www.princeton.edu/~mudd/thesis/Submissionguide.pdf
        % ProQuest requires a double spaced version (set previously). They will take an electronic copy, so we want links in the pdf, but also copies may be printed or made into microfilm in black and white, so we want outlined links instead of colored links.
        \usepackage{hyperref}
        \hypersetup{bookmarksnumbered}
        % copy the already-set title and author to use in the pdf properties
        \makeatletter
        \hypersetup{pdftitle=\@title,pdfauthor=\@author}
        \makeatother
    \else
        % Online copy
        % adds internal linked references, pdf bookmarks, etc
        % turn all references and citations into hyperlinks:
        %  -- not for printed copies
        % -- automatically includes url package
        % options:
        %   colorlinks makes links by coloring the text instead of putting a rectangle around the text.
        \usepackage{hyperref}
        \hypersetup{colorlinks,bookmarksnumbered}
        % copy the already-set title and author to use in the pdf properties
        \makeatletter
        \hypersetup{pdftitle=\@title,pdfauthor=\@author}
        \makeatother
        % make the page number rather than the text be the link for ToC entries
        %\hypersetup{linktocpage}
    \fi % proquest or online formatting
\fi % printed or online formatting

%%%%%%%%%%%%%%%%%%%%%%%%%%%%%%%%%%%%%%%%%%%%%%%%%%%%%%%%%%%%%%%%%%%%%%%%%%%%%%%%%%%%%%%%%%%%%%%%
%%%% Define commands

%Highlight notes for future edits to the thesis
%\newcommand{\todo}[1]{\textbf{\emph{TODO:}#1}}

%create an environment that will indent text
%\raggedright makes them left aligned instead of justified

\newenvironment{indenttext}{
    \begin{list}{}{ \itemsep 0in \itemindent 0in
    \labelsep 0in \labelwidth 0in
    \listparindent 0in
    \topsep 0in \partopsep 0in \parskip 0in \parsep 0in
    \leftmargin 1em \rightmargin 0in
    \raggedright
    }
    \item
  }
  {\end{list}}

% another environment that's an indented list, with no spaces between items -- if we want multiple items/lines. Useful in tables. Use \item inside the environment.
%   \raggedright makes them left aligned instead of justified

\newenvironment{indentlist}{
    \begin{list}{}{ \itemsep 0in \itemindent 0in
    \labelsep 0in \labelwidth 0in
    \listparindent 0in
    \topsep 0in \partopsep 0in \parskip 0in \parsep 0in
    \leftmargin 1em \rightmargin 0in
    \raggedright
    }
  }
  {\end{list}}

\newcommand{\ket}[1]{|#1\rangle}
\newcommand{\tket}[1]{$|#1\rangle$}
\newcommand{\bra}[1]{\langle#1|}
\newcommand{\tbra}[1]{$\langle#1|$}
\newcommand{\innerp}[2]{\langle#1|#2\rangle}
\newcommand{\tinnerp}[2]{$\langle#1|#2\rangle$}
\newcommand{\outerp}[2]{\q{#1}\qa{#2}}
\newcommand{\touterp}[2]{\tq{#1}\tqa{#2}}
\newcommand{\beq}{\begin{equation}}
\newcommand{\eeq}{\end{equation}}
\newcommand{\myfig}[4][width=\textwidth]{\begin{figure}\begin{center}\includegraphics[#1]{#2}\caption{#3}\label{#4}\end{center}\end{figure}}
\renewcommand{\exp}[1]{\text{exp}\left(#1\right)}

%%%%%%%%%%%%%%%%%%%%%%%%%%%%%%%%%%%%%%%%%%%%%%%%%%%%%%%%%%%%%%%%%%%%%%%%%%%%%%%%%%%%%%%%%%%%%%%%
%%%% Front-matter
% For early drafts, you may want to disable some of the frontmatter. Simply change this to "\ifodd 1" to do so.
\ifodd 1
    % front-matter disabled while writing chapters
    \renewcommand{\maketitlepage}{}
    \renewcommand*{\makecopyrightpage}{}
    \renewcommand*{\makeabstract}{}
    % you can just skip the \acknowledgements and \dedication commands to leave out these sections.
\else
    \abstract{
    % Abstract can be any length, but should be max 350 words for a Dissertation for ProQuest's print indicies (150 words for a Master's Thesis) or it will be truncated for those uses.
        Starting with Richard Feynman's suggestion that quantum mechanical computers may be able to efficiently simulate quantum mechanical systems \cite{historyofqc}, there has been a tremendous amount of effort worldwide to create scalable quantum computers. Various experts have analyzed possible implementation schemes for quantum computers. Much like the time when classical computers had several proposed schemes, quantum computers have seen many attractive candidates for their implementation.

Here, we introduce quantum architecture via quantum gates and circuits that result from their combinations. We investigate the mathematical models of these quantum gates and discuss the requirements for scalable quantum circuits. We find universal sets of logic gates that we then implement using optical devices that are commonly available in a laboratory setting. In doing so, we develop the analytical skills necessary for determining the feasibility of quantum architecture schemes. 

Finally, we discuss the drawbacks of the single photon quantum computing scheme and discuss possible areas of research using this scheme.

    }
    \acknowledgements{
        The Manhattan College community has welcomed me with open arms and provided an environment where I was able to discover my academic interests. For this, I am very grateful for this, and for all the support given to me by my friends. 

My attendance at Manhattan College would not have been possible without the generosity of Mr. and Mrs. O'Malley, to whom I am forever indebted. 

As the first engineer in my family, I relied heavily on advice from all the professors in the Department of Electrical and Computer Engineering at Manhattan College. They have gone beyond providing academic mentorship to nurture my varying interests. In particular, Prof. Nisteruk has witnessed my interests as they changed from signals and systems to linear systems, and finally to quantum computing. I am grateful to him for encouraging me to work outside my comfort levels until I had something to be proud of. His frequent lessons of quantum mechanics outside my regular academic coursework have broadened my perspectives beyond the principles of electrical engineering.

Despite my major, I have always received support from the Department of Mathematics and Computer Science. Several professors in the department took the time to give me independent study courses. I am thankful to Prof.\ Bishop, Prof.\ Boothe, Prof.\ Goldstone, Prof.\ Jura and Prof.\ McCabe for their patience and support. In particular, Prof.\ Bishop and Prof.\ Boothe have nurtured my interest in mathematics and computer science through two rewarding research projects that have helped shape my career significantly.

I also want to acknowledge the support provided to me by the Department of Physics, and particularly by Prof. Liby. He took on the role of advising me willingly and filled the gaps in my knowledge of physical optics. His mentorship has prepared me well for my upcoming graduate studies.

I would like to thank my entire family for their continued support. My academic growth has always been supplemented by their wisdom. In particular, I would like to thank my brother Kidus for constantly reviewing anything I sent his way and my mother Hiwot for the emotional support that she has given me throughout my life at and away from home.

I would also like to thank my girlfriend Mahlet for being a constant source of motivation and encouragement.

Last but certainly not least, I would like to thank God for creating me and allowing me to delve into an exploration of the computing capabilities of quantum phenomena afforded by nature.

    }

    \dedication{To R.P. Feynman}
\fi  % disable frontmatter

%%%%%%%%%%%%%%%%%%%%%%%%%%%%%%%%%%%%%%%%%%%%%%%%%%%%%%%%%%%%%%%%%%%%%%%%%%%%%%%%%%%%%%%%%%%%%%%%
%%%% Hide some chapters
%%% If you want to produce a pdf that includes only certain chapters, specify them with includeonly, in addition to including all chapters below.
%\includeonly{ch-intro/chapter-intro}
%%% You can also specify multiple chapters.
%\includeonly{ch-intro/chapter-intro,ch-usage/chapter-usage}
%\includeonly{chap1,chap2,chap3}

%%%%%%%%%%%%%%%%%%%%%%%%%%%%%%%%%%%%%%%%%%%%%%%%%%%%%%%%%%%%%%%%%%%%%%%%%%%%%%%%%%%%%%%%%%%%%%%%
%%%% Notes:
% Footnotes should be placed after punctuation.\footnote{place here.}
% Generally, place citations before the period~\cite{anotherauthor}.
% The proper usage for i.e., and e.g., include commas ``(e.g., option A, option B)''

%%%%%%%%%%%%%%%%%%%%%%%%%%%%%%%%%%%%%%%%%%%%%%%%%%%%%%%%%%%%%%%%%%%%%%%%%%%%%%%%%%%%%%%%%%%%%%%%
%%%% Import chapters
\begin{document}

\makefrontmatter

\chapter{Quantum circuits\label{ch:qcirc}}

\section{The qubit}

\section{Bloch sphere interpretation of the qubit}

\section{Single qubit operations}
Operations on a qubit must preserve the norm of the qubit, i.e., given an operation $O$ on a single qubit and two qubits $\ket{\psi} = a\ket{0} + b\ket{1}$ and $\ket{\psi'} = O\ket{\psi} = a'\ket{0} + b'\ket{1}$, the normalization conditions
\beq
a^2 + b^2 = a'^2 + b'^2 = 1
\eeq
must hold. For this reason, operators on single qubits are 2x2 unitary matrices. 

The most common single qubit operations are represented by the Pauli matrices.  The Pauli matrices are shown below. 

\beq
X \equiv \left[\begin{array}{cc}0 & 1\\
1 & 0\end{array}\right] \text{ ; } Y \equiv \left[\begin{array}{cc}0 & -i\\
i & 0\end{array}\right] \text{ ; } Z \equiv \left[\begin{array}{cc}1 & 0\\
0 & -1\end{array}\right]
\eeq

Three other matrices that are commonly used in quantum computing are the Hadamard ($H$), $\pi/8$ ($T$) and phase ($S$) gates. These are shown below.
\beq
H \equiv \frac{X+Z}{\sqrt{2}} = \frac{1}{\sqrt{2}}\left[\begin{array}{cc}1 & 1\\
1 & -1\end{array}\right] \text{ ; } T \equiv \left[\begin{array}{cc}1 & 0\\
0 & exp(i\pi/4)\end{array}\right] \text{ ; } S \equiv T^2 = \left[\begin{array}{cc}1 & 0\\
0 & i\end{array}\right]
\eeq

In Appendix \ref{ch:expmtx}, we have shown how to exponentiate matrices. Using these results, we now introduce three additional unitary matrices known as \textit{rotation matrices} corresponding to the Pauli matrices. These are shown below.

\begin{align}
R_x(\theta) &\equiv \exp{-i\theta X/2} = \cos\left(\frac{\theta}{2}\right)I - i\sin\left(\frac{\theta}{2}\right)X = 
\left[\begin{array}{cc}\cos\frac{\theta}{2} & -i\sin\frac{\theta}{2}\\
-i\sin\frac{\theta}{2} & \cos\frac{\theta}{2}\end{array}\right]\\ 
R_y(\theta) &\equiv \exp{-i\theta Y/2} = \cos\left(\frac{\theta}{2}\right)I - i\sin\left(\frac{\theta}{2}\right)Y = 
\left[\begin{array}{cc}\cos\frac{\theta}{2} & -\sin\frac{\theta}{2}\\
\sin\frac{\theta}{2} & \cos\frac{\theta}{2}\end{array}\right]\\ 
R_z(\theta) &\equiv \exp{-i\theta Z/2} = \cos\left(\frac{\theta}{2}\right)I - i\sin\left(\frac{\theta}{2}\right)Z = 
\left[\begin{array}{cc}\exp{-i\theta/2} & 0\\
0 & \exp{i\theta/2}\end{array}\right]
\end{align}

In general, the rotation by $\theta$ about an axis defined by the real unit vector $\hat{n} = (n_x,n_y,n_z)$ is applied using the following matrix.
\beq
R_{\hat{n}}(\theta) \equiv \exp{-i\theta \frac{n_xX + n_yY + n_zZ}{2}} = \cos\left(\frac{\theta}{2}\right)I - i\sin\left(\frac{\theta}{2}\right)(n_xX + n_yY + n_zZ)
\eeq

Without proof, we present a very useful way of representing unitary operator matrices below. Any unitary operator $U$ can be represented by a matrix which is a product of rotations in the y and z axes plus a global phase. Interested readers are pointed to reference \cite{nielsen2000} for a detailed proof.
\beq
U = \exp{i\alpha}R_z(\beta)R_y(\gamma)R_z(\delta)
\eeq

Yet another representation of a unitary operator matrix which follows from above is shown here without proof. Interested readers are pointed to reference \cite{nielsen2000} for a detailed proof.
\beq
U = \exp{i\alpha}AXBXC
\label{eq:unitarytosingle}
\eeq
In the above representation, $A,B$ and $C$ are unitary themselves and $ABC = I$.
These representations will be important as we define controlled operations using multiple qubits in the next section. One additional set of identities that we need to keep in mind is the following. The proofs for these are straightforward substitutions and are not shown here.
\beq
HXH = Z \text{ ; } HYH = -Y \text{ ; } HZH = X
\eeq

Figure \ref{fig:singlequbitgates} summarizes all the single qubit gates and shows their circuit notations.

\myfig[scale=0.50]{qcirc/fig/singlequbitgates.png}{Single-qubit gates and their circuit notations}{fig:singlequbitgates}

\section{Multiple qubits and controlled operations}
So far, we have talked about operations on single qubits. We now discuss a class of families on multiple qubits known as \textit{controlled operations}. In particular, we will discuss two-qubit controlled operations. These discussions can intuitively be generalized for larger numbers of qubits.

The simplest controlled operation is the controlled version of the classical \textsc{NOT} gate, known as the \textsc{CNOT} gate. It operates on two qubits known as the \textit{control qubit} and the \textit{target qubit}. Its operation is described as follows: if the control qubit is set (to \tket{1}), then the target qubit is inverted (from \tket{0} to \tket{1} and vice versa). The shorthand notation for the \textsc{CNOT} gate is given below.

\beq
\text{\textsc{CNOT}: } \ket{control}\ket{target} \rightarrow \ket{control}\ket{control \oplus target}
\eeq

The matrix representation for the \textsc{CNOT} operator can be determined as follows from the above shorthand notation.
\begin{align}
\text{\textsc{CNOT}: } \ket{0}\ket{0} &\rightarrow \ket{0}\ket{0}\\
\text{\textsc{CNOT}: } \ket{0}\ket{1} &\rightarrow \ket{0}\ket{1}\\
\text{\textsc{CNOT}: } \ket{1}\ket{0} &\rightarrow \ket{1}\ket{1}\\
\text{\textsc{CNOT}: } \ket{1}\ket{1} &\rightarrow \ket{1}\ket{0}
\end{align}

In the computational basis \{\tket{0}, \tket{1}\}, the \textsc{CNOT} gate acts on two qubits and is therefore 4x4. The columns of the \textsc{CNOT} matrix are the outputs of \tket{0}\tket{0}, \tket{0}\tket{1}, \tket{1}\tket{0} and \tket{1}\tket{1} respectively. Hence,

\beq
\textsc{CNOT} \equiv \left[\begin{array}{cccc}1 & 0 & 0 & 0\\
0 & 1 & 0 & 0\\
0 & 0 & 0 & 1\\
0 & 0 & 1 & 0\end{array}\right]
\eeq

\myfig[width=3cm]{qcirc/fig/cnot.png}{Circuit notation for a \textsc{CNOT} gate}{fig:cnot}

The circuit notation for the \textsc{CNOT} operator is shown in figure \ref{fig:cnot}. In general, a controlled-U operator can be represented by a matrix by noting how it affects combinations of input qubits. The shorthand notation is shown below.

\beq
\text{controlled-U: } \ket{control}\ket{target} \rightarrow \ket{control}U^{control}\ket{target}
\eeq

This general notation leads to a general form for the matrix of a controlled-U operator, which is shown below.

\beq
\text{controlled-U} \equiv \left[\begin{array}{cc}I & 0\\
0 & U\end{array}\right]
\eeq

Note that the above matrix is 4x4 because both $I$ and $U$ are 2x2. Also note that the \textsc{CNOT} operator has the same matrix representation as a controlled-X operator. Notationally,

\beq
\textsc{CNOT} \equiv \left[\begin{array}{cccc}1 & 0 & 0 & 0\\
0 & 1 & 0 & 0\\
0 & 0 & 0 & 1\\
0 & 0 & 1 & 0\end{array}\right] = \left[\begin{array}{cc}I & 0\\
0 & X\end{array}\right]
\eeq

The commonly used circuit notation for a controlled-U operation is shown in figure \ref{fig:cugate}.

\myfig{qcirc/fig/controlledu.png}{A controlled-U operator. Note the control and target qubits}{fig:cugate}

Recall from equation \ref{eq:unitarytosingle} that we can represent unitary operators by elementary operators. Therefore, for a controlled-U gate, we have the following: when the control is disabled, the identity matrix is applied to the target qubit. Otherwise, $U = \exp{i\alpha}AXBXC$ is applied to the target qubit. Also recalling the constraints on $A,B$ and $C$ such that $ABC = I$, this implies that the control qubit affects the $X$ operators, turning them into \textsc{CNOT} gates. 

Also note the following.
\begin{align}
\exp{i\alpha}\text{: } \ket{0}\ket{0} &\rightarrow \ket{0}\ket{0}\\
\exp{i\alpha}\text{: } \ket{0}\ket{1} &\rightarrow \ket{0}\ket{1}\\
\exp{i\alpha}\text{: } \ket{1}\ket{0} &\rightarrow \ket{1}\exp{i\alpha}(\ket{0}) = \exp{i\alpha}(\ket{1})\ket{0}\\
\exp{i\alpha}\text{: } \ket{1}\ket{1} &\rightarrow \ket{1}\exp{i\alpha}(\ket{1}) = \exp{i\alpha}(\ket{1})\ket{1}
\end{align}

From the above equations, we observe that the effect of the global phase can be applied to the control or target qubits. Since it only applies when the control qubit is set, we will apply it there, giving the circuit diagram in figure \ref{fig:udecomp}.

\myfig{qcirc/fig/udecomp.png}{A controlled-U gate decomposed into elementary single-qubit gates}{fig:udecomp}

The above discussion easily applies when there are multiple control and target qubits. Given a unitary operator $U$ applied on $n$ control qubits \{$x_1,x_2,\ldots,x_n$\} and $k$ target qubits \{$y_1,y_2,\dots,y_k$\}, we have the following shorthand notation.

\begin{align}
\text{controlled-U: } \ket{x_1}\ket{x_2}\ldots\ket{x_{n-1}}\ket{x_n}\ket{y_1}\ket{y_2}\ldots\ket{y_{k-1}}\ket{y_k} \rightarrow\\ \ket{x_1}\ket{x_2}\ldots\ket{x_{n-1}}\ket{x_n}U^{x_1x_2\ldots x_{n-1}x_n}\ket{y_1}\ket{y_2}\ldots\ket{y_{k-1}}\ket{y_k}
\end{align}

This shorthand notation is shown in circuit notation in figure \ref{fig:cugatem}.

\myfig[height=8cm,width=6cm]{qcirc/fig/controlledumultiple.png}{A multiple-control multiple-target controlled-U operator with 4 control and 3 target qubits}{fig:cugatem}

\section{Summary}

In this chapter, we have introduced the two-state quantum system used in quantum computation known as the qubit. We have indicated its wavefunction and used the Bloch sphere as a means of visualizing qubits. The Bloch sphere is particularly important when we consider single qubit gates. It helps us visualize the effects of the unitary operators that we normally encounter in matrix form.

We have also discussed the fundamental single-qubit gates -- the Pauli gates, their derived rotation operators, the Hadamard, phase and $\pi/8$ gates.

After introducing single-qubit gates, we showed how they can be controlled using additional qubits to form multiple-qubit gates. We showed how to derive the matrices corresponding to these operators and derived general controlled operations in terms of elementary single-qubit gates. The method of control was shown to be applicable even when the number of control and target qubits increases.

For additional reading, including topics such as universal gates and operator approximations, the interested reader is referred to chapter 4 of \cite{nielsen2000}.

\chapter{Optical Implementations\label{ch:optimp}}


\appendix % all chapters following will be labeled as appendices
\chapter{Exponentiating Matrices\label{ch:expmtx}}

In this appendix, we prove the following.
\beq
\exp{iAx} = cos(x)I + isin(x)A
\eeq
for a real number $x$ and matrix $A$ such that $A^2 = I$ and $A^0 = I$. Recall the power series expansion for the exponential $\exp{x}$ for all $x$.

\beq
\exp{x} = \sum\limits_{n=0}^{\infty} \frac{x^n}{n!}
\eeq
We now rewrite this power series expansion for $\exp{x}$ after replacing $x$ by $iAx$.

\begin{align}
\exp{iAx} &= \sum\limits_{n=0}^{\infty}\frac{\left(iAx\right)^n}{n!}\\
&= \frac{I}{0!} + \frac{iAx}{1!} + \frac{\left(iAx\right)^2}{2!} + \frac{\left(iAx\right)^3}{3!} + \ldots\\
&= \frac{A^2}{0!} + \frac{iAx}{1!} + \frac{\left(iAx\right)^2}{2!} + \frac{\left(iAx\right)^3}{3!} + \ldots
\end{align}
Noting that even powers of $A$ reduce to identity and $i^2 = -1$, we now rearrange the terms in the above equation as follows.

\begin{align}
\exp{iAx} &= \frac{A^2}{0!} + \frac{iAx}{1!} + \frac{\left(iAx\right)^2}{2!} + \frac{\left(iAx\right)^3}{3!} + \ldots\\
&= \left(\frac{x^0}{0!} - \frac{x^2}{2!} + \frac{x^4}{4!} - \ldots \right)I + i\left(\frac{x^1}{1!} - \frac{x^3}{3!} + \frac{x^5}{5!} - \ldots \right)A\\
\end{align}
The power series expansions for the sine and cosine functions appear in the above equation. We will state the power series expansions for these two functions below.

\begin{align}
sin(x) &= \sum\limits_{n=0}^{\infty} \left(-1\right)^n\frac{x^{2n+1}}{\left(2n+1\right)!} = \left(\frac{x^1}{1!} - \frac{x^3}{3!} + \frac{x^5}{5!} - \ldots \right)\\
cos(x) &= \sum\limits_{n=0}^{\infty} \left(-1\right)^n\frac{x^{2n}}{\left(2n\right)!} = \left(\frac{x^0}{0!} - \frac{x^2}{2!} + \frac{x^4}{4!} - \ldots \right)
\end{align}
Therefore,

\begin{align}
\exp{iAx} &= \left(\frac{x^0}{0!} - \frac{x^2}{2!} + \frac{x^4}{4!} - \ldots \right)I + i\left(\frac{x^1}{1!} - \frac{x^3}{3!} + \frac{x^5}{5!} - \ldots \right)A\\
&= cos(x)I + isin(x)A\qed
\end{align}


%\include{appendicies/}

\singlespacing
\bibliographystyle{plain}
\cleardoublepage
\ifdefined\phantomsection
    \phantomsection
\else
\fi
\addcontentsline{toc}{chapter}{Bibliography}
\bibliography{thesis}
\end{document}
