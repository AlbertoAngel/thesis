Starting with Richard Feynman's suggestion that quantum mechanical computers may be able to efficiently simulate quantum mechanical systems \cite{historyofqc}, there has been a tremendous amount of effort worldwide to create scalable quantum computers. Various experts have analyzed possible implementation schemes for quantum computers. Much like the time when classical computers had several proposed schemes, quantum computers have seen many attractive candidates for their implementation.

Here, we introduce quantum architecture via quantum gates and circuits that result from their combinations. We investigate the mathematical models of these quantum gates and discuss the requirements for scalable quantum circuits. We find universal sets of logic gates that we then implement using optical devices that are commonly available in a laboratory setting. In doing so, we develop the analytical skills necessary for determining the feasibility of quantum architecture schemes. 

Finally, we discuss the drawbacks of the single photon quantum computing scheme and discuss possible areas of research using this scheme.
